% !TEX root = ../gamification-in-human-resource-management.tex
% !TeX program = xelatex
\chapter{مقدمه}
\section{پیش‌گفتار}
زمانی که بتوانیم خشکی محیط کار را از بین ببریم، می‌توانیم انگیزه‌های فراوانی در کارمندان و منابع انسانی یک سازمان ایجاد کنیم؛ بازی‌وارسازی\LTRfootnote{Gamification} به دنبال ایجاد همین قابلیت است. بازی‌وارسازی در اصل یک علم برای ایجاد روندهای بازی در دنیای واقعی است \cite{boudlaie}؛ به طور دقیق‌تر بازی‌وارسازی روشی است که در آن با استفاده از عناصر و المان‌های موجود در بازی‌ها، دنیای واقعی را مشابه یک بازی در نظر می‌گیریم تا بتوانیم تجربه‌ای که بازی‌ها ایجاد می‌کنند را در دنیای واقعی نیز به دست آوریم.

بازی‌وارسازی در صنایع و کاربردهای مختلفی قابل استفاده است؛ کاربردهایی مثل مدیریت منابع انسانی، آموزش، بازاریابی، بهبود تجربه مشتری و غیره. در این مقاله به طور ویژه به کاربرد بازی‌وارسازی در مدیریت منابع انسانی پرداخته می‌شود.
\section{مفاهیم پایه}
برای آنکه بتوانیم دید بهتری نسبت به موضوع پیدا کنیم، باید بعضی از مفاهیم پایه موجود در مدیریت منابع انسانی و بازی‌وارسازی مطرح شوند.

برای بررسی کل فرایند مدیریت منابع انسانی باید پنج جنبه استخدام، مدیریت عمل‌کرد و مشارکت، آموزش و استعداد، مدیریت دانش و فرهنگ سازمانی مورد واکاوی قرار گیرند \cite{amiriamin}. در این‌صورت می‌توانیم با بررسی نقش بازی‌وارسازی در هر کدام از این جنبه‌ها به کاربرد نهایی بازی‌وارسازی در مدیریت منابع انسانی بپردازیم.

همانطور که حاتمی \cite{atoz} اشاره می‌کند، در یک بازی، ما همیشه با نقش بازیکن برخورد داریم، به همین خاطر استفاده از لفظ \emph{کاربر} در مورد کسانی که می‌خواهیم برایشان بازی طراحی کنیم، درست نمی‌باشد؛ در ادامه نوشته به خاطر قراردادن مخاطب در دنیای بازی، از لفظ \emph{بازیکن} استفاده می‌شود. برای ایجاد بازی‌وارسازی در هر فرایندی، کار اصلی و نهایی ما طراحی بازی‌هایی برای بازیکن‌هاست.

بنابر مطالعات حاتمی \cite{atoz} بازیکن‌ها دسته‌بندی‌های مختلفی دارند. دسته‌بندی بازیکنان به این شکل است:
\subparagraph{دستاوردگرا}
این دسته از بازیکنان کسانی هستند که به دنبال ثبت رکورد، افزایش رتبه در جدول رده‌بندی و دریافت مدال‌ها و نشان‌های افتخار هستند و با بازیکنان دیگر کاری ندارند. به طور کلی این افراد شیفته پیروزی و برنده‌شدن هستند. برای مثال مدیران سازمان در این دسته قرار می‌گیرند.
\subparagraph{جستجوگر}‫
این افراد به دنبال کشف بازی، یافتن نقاط کور، پیداکردن جاذبه‌ها و رمزهای بازی و به طور کلی جستجوی در بازی حرکت می‌کنند. برای مثال افراد واحد تحقیق‌وتوسعه می‌توانند از این دسته باشند.
\subparagraph{معاشرت‌گرا}‫
کسانی هستند که جنبه اجتماعی بازی برای آن‌ها مهم است و بازی می‌کنند تا با دیگران ارتباط برقرار کنند. به نوعی این افراد از بازی برای یافتن نقاط مشترک خود با دیگران استفاده می‌کنند و چیزی که بازی را برای آن‌ها جذاب می‌کند خوش‌گذرانی با دیگران است. در یک سازمان کارمندان زیادی می‌توانند در این دسته قرار گیرند و چندان به موقعیت کاری مربوط نیست.
\subparagraph{قاتل}‫
قاتل‌ها کسانی هستند که به دنبال تسلط بر دیگران هستند. یک قاتل از پیروزی خود و شکست دیگری خوشحال می‌شود. رهبران سازمان در این دسته قرار می‌گیرند.

برای اینکه متوجه شویم هر کاربرد موجود در دنیای ما به چه بازی و فرایند بازی‌گونه‌ای نیاز دارد، باید ابتدا بازیکن‌های مربوط به آن کاربرد را شناسایی کنیم، مشخص کنیم از چه دسته‌ای هستند و یک بازی متناسب با دسته آن‌ها طراحی کنیم. علاوه بر نگاه به سازمان باید به جامعه‌ای که سازمان در آن قرار دارد نیز توجه کرد. در ایران دسته‌های قاتل و جستجوگر درصد بالایی از بازیکنان را در بر می‌گیرند، در حالی که در غرب، معاشرت‌گرایان کثرت دارند و حدودا ۸۰ درصد جامعه را شامل می‌شوند \cite{atoz}.

طبق دسته‌بندی حاتمی \cite{atoz}، بازی و بازی‌وارسازی نیز انواع مختلفی دارد که در ادامه به آن‌ها می‌پردازیم.
\subparagraph{پنهان}‫
در این مدل، بازیکن متوجه نمی‌شود که در حال انجام یک بازی است. در حقیقت بازی از نگاه بازیکن مخفی شده‌است.
\subparagraph{ظاهر}‫
در این حالت بازیکن می‌داند که به انجام یک بازی می‌پردازد و نسبت به بازی عالم است.
\subparagraph{فیزیکی}‫
یک بازی (مثل بازی‌های رومیزی) می‌تواند بر مبنای عناصر فیزیکی طراحی شود و عناصر بازی قابل لمس باشند که به این بازی‌ها، بازی‌های فیزیکی گفته می‌شود. زمانی که در بازی‌وارسازی از بازی‌های فیزیکی استفاده شود، مدل فیزیکی رخ می‌دهد.
\subparagraph{دیجیتالی}‫
زمانی که بازی‌ها بر مبنای اپلیکیشن یا محتوای دیجیتالی توسعه داده شوند، یک بازی دیجیتالی ساخته می‌شود که با استفاده از این بازی‌ها، مدل بازی‌وارسازی دیجیتالی به وجود می‌آید.

همچنین دو نوع دیگر نیز قابل ذکر است \cite{killer}:

\subparagraph{موردی}‫
در مدل موردی تنها در قسمت‌های خاصی از کاربرد مورد نظر از بازی‌وارسازی استفاده می‌شود.
\subparagraph{کامل}‫
در مدل کامل کل فضای کاربرد تحت تاثیر بازی‌وارسازی قرار می‌گیرد.

با شناخت انواع بازی و بازی‌وارسازی می‌توانیم به نحو بهینه‌ای از روش‌های بازی‌وارسازی در کاربردهای مختلف استفاده کنیم.

\section{بازی‌وارسازی در منابع انسانی}
در فرایند مدیریت منابع انسانی فعلی اشکالات بسیاری وجود دارد و در سازمان‌ها از هر روشی که بتواند این مشکلات را حل کند، استقبال می‌شود. استخدام‌های سازمان‌ها به وضعیت مطلوب نزدیک نیست و بعضی از استخدام‌ها منجر به به‌کارگیری افراد ناشایست می‌شوند. کارمندان در محیط شرکت کسل هستند و انگیزه‌ای برای انجام کارها ندارند. جلسات به صورت بی‌نتیجه برگزار می‌شوند و اتلاف وقت بسیاری دارند. خود کارمندان نیز در زندگی خود با بیهودگی، افسردگی و پوچی در معنای زندگی مواجه هستند که بخشی از این مشکلات شخصی به خاطر محیط کاری نامطلوب به وجود می‌آیند. بازی‌وارسازی در مورد این مشکلات می‌تواند کمک‌کننده باشد و به این وسیله می‌توان محیط کاری را بهبود بخشید \cite{boudlaie}.

البته طبق مطالعات سایت آقای گیمیفیکیشن \cite{scenarios} در استفاده از بازی‌وارسازی در محیط کاری، نگرانی‌هایی هم وجود دارد؛ مواردی مثل اینکه کارمندان درگیر بازی‌های بچگانه می‌شوند و بازی‌کردن برای چنین افرادی مناسب نیست! نکته اصلی در این رابطه این است که بازی‌وارسازی لزوما به معنی بازی‌کردن نیست؛ می‌توان صرفا از المان‌های بازی استفاده کرد \cite{poliver}. در فرهنگ متداول محیط کاری، متضاد کار، بازی است و برعکس؛ چنان‌که می‌شنویم افراد در گفتگوهای خود هنگامی که می‌خواهند بگویند شخصی از زیر کار فرار می‌کند، می‌گویند بازی می‌کند. بنا بر همین نوع نگاه، مدیران نگران هستند که هنگام استفاده از بازی‌وارسازی، کارمندان از محیط کار دور شوند و به تبع بهره‌وری آن‌ها کاهش یابد؛ اما این نوع نگاه صحیح نیست. در حقیقت متضاد کار و همچنین متضاد بازی، افسردگی است \cite{tedx}. در حقیقت کسی که کار یا بازی نمی‌کند، افسرده است و کسی که افسرده است نمی‌تواند بازی کند یا کار کند. در بازی‌وارسازی صرفا از بازی‌ها استفاده می‌شود تا کار جذاب شود نه اینکه افراد فعالیت اصلی خود را بازی‌کردن ببینند.

در بازی‌وارسازی باید به این نکته توجه کرد که از این روش باید با هدف افزایش بهره‌وری افراد طوری که خود افراد تمایل دارند، استفاده شود \cite{tedx}. در حقیقت بازی‌وارسازی نباید برای ایجاد نوعی اجبار برای کارمندان یا دیکته‌کردن موارد به آن‌ها استفاده شود تا افراد طوری که ما می‌خواهیم رفتار کنند!
نکته دیگر این است که اگرچه بازی‌وارسازی در مدیریت منابع انسانی می‌تواند کمک‌کننده باشد اما خطراتی هم دارد \cite{atoz,reghmosh}. ممکن است کارمندان نسبت به انجام بازی‌ها مشتاق نباشند و نسبت به تغییرات جبهه‌گیری کنند. در بسیاری از پیاده‌سازی‌های بازی‌وارسازی از تکنیک ارائه پاداش استفاده می‌شود در حالی که این تکنیک لزوما کاربردی نیست؛ چون گاهی کارمند به دنبال پاداش نیست \cite{scenarios}! بنابر نظر حاتمی \cite{atoz} اگر از روش‌های بازی‌وارسازی به شکل مطلوب استفاده نشود، صدمات جبران‌ناپذیری به منابع انسانی سازمان وارد می‌شود؛ مثلا، یک پیاده‌سازی غلط از بازی‌وارسازی در سازمان‌ها برای پرداخت بی‌رویه پاداش یا اصلا استفاده غیر منطقی از المان‌های بازی‌وارسازی، می‌تواند منجر به افزایش توقع کارمندان شود که این مسئله در بلندمدت به خاطر عدم توانایی سازمان در برآورده‌کردن توقعات، انگیزه کارمندان نسبت به کار را حتی بیشتر از قبل کاهش می‌دهد. یک نمونه دیگر از پیاده‌سازی غلط بازی‌وارسازی هنگامی رخ می‌دهد که طراحان بازی، کارمندان را همانند کودکان در نظر می‌گیرند و شناخت درستی نسبت به جامعه بازیکنان ندارند که در این حالت بازی‌وارسازی نتیجه‌ای کاملا برعکس دارد.

\subsection{بازی فوتبال}
یکی از مثال‌ها برای توضیح بازی‌وارسازی در فرایند مدیریت منابع انسانی، بازی فوتبال است. در صنعت فوتبال بازیکن‌ها (ورزشکاران) عملا کارمندان یک باشگاه ورزشی هستند و فوتبال بازی‌کردن، کار آن‌ها حساب می‌شود. کار این کارمندان می‌تواند برای باشگاه امتیازاتی مثل جوایز مسابقات یا صعود در جدول رده‌بندی به ارمغان آورد.
به خاطر ماهیت این کار که در صنعت سرگرمی فعالیت می‌کند و بیشتر جنبه تفریحی دارد، بازی‌وارسازی به خوبی در این کار می‌تواند پیاده‌سازی شود و نتایج آن مشخص شود. چند مورد از پیاده‌سازی‌های مرتبط با عناصر بازی‌وارسازی در بازی فوتبال در ادامه بیان می‌شوند.
\subparagraph{بازخورد\protect\LTRfootnote{Feedback}}
بازخورد یکی از عناصر بازی‌وارسازی است که در آن به بازیکن به شکل آنی بازخوردهایی ارائه می‌شود و او را از نتیجه تصمیماتش آگاه می‌کند \cite{atoz}. در بازی فوتبال پس از اتمام هر بازی، بازیکن‌ها از سمت مربیان و کارشناسان مورد ارزیابی قرار می‌گیرند و نمره‌هایی به آن‌ها اختصاص داده می‌شود یا به شکلی دیگر آمارهای مرتبط با بازیکن (مثل تعداد پاس‌های صحیح یا درصد خطا در بازپس‌گیری توپ) به اطلاع وی رسانده می‌شود. این روند نمره‌دهی و ارائه آمار که پیاده‌سازی عنصر بازخورد از مجموعه عناصر بازی‌وارسازی است، به بازیکنان کمک می‌کند از نتیجه تمرینات خود مطلع شوند، ضعف‌های خود را بیابند و بتوانند بهتر بازی کنند و از سمت دیگر امتیازدهی می‌تواند نقش مشوق را برای بازیکنان ایفا کند. به این شیوه بازیکن روش کارکردن بیشتر و بهتر را متوجه می‌شود و بهره‌وری وی افزایش می‌یابد.
\subparagraph{جدول برندگان\protect\LTRfootnote{Leader board}}
جدول برندگان عنصری از بازی‌وارسازی است که بر اساس امتیازات بازیکنان به آن‌ها رتبه‌ای در جدول می‌دهد و به این شکل بازیکن می‌تواند خود را با دیگر بازیکنان مقایسه کند \cite{atoz}. در یک بازی فوتبال، انتخاب بازیکن هفته، ماه و سال، یکی از پیاده‌سازی‌های مرتبط با جدول برندگان است.
\subparagraph{تصمیم‌گیری\protect\LTRfootnote{Choice}}
در یک بازی دو نوع هدف بلندمدت و کوتاه‌مدت وجود دارد که در هدف کوتاه‌مدت بازیکن سعی می‌کند آمارهای آنی خود را بهبود ببخشد در هدف بلندمدت بازیکن باید به یک نتیجه نهایی برسد. عنصر تصمیم‌گیری به این می‌پردازد که بازیکن در هر لحظه چگونه تصمیم‌گیری می‌کند و انتخاب او بین رسیدن به هدف بلندمدت و هدف کوتاه‌مدت کدامیک خواهد بود که گاهی این دو هدف در تعارض با یکدیگر قرار می‌گیرند \cite{atoz}. در بازی باید انتخاب‌های معناداری وجود داشته باشند تا بازیکن با تصمیم‌گیری درگیر شود و همین کار باعث ایجاد جذابیت برای بازی می‌شود. آمارهای شخصی بازیکنان و انتخاب به عنوان بازیکن برتر زمین به عنوان هدف‌های کوتاه‌مدت در مقابل مشارکت با دیگر بازیکنان برای بردن جام‌های مختلف توسط تیم و باشگاه، به عنوان هدف‌های بلند مدت می‌توانند این انتخاب و تصمیم‌گیری را در بازی فوتبال ایجاد کنند.
\subparagraph{چالش\protect\LTRfootnote{Challenge}}
عنصر چالش اشاره به چالش‌هایی دارد که بازی برای بازیکن تعیین می‌کند \cite{atoz}. مثلا دریبل‌زدن یا گل از طریق ضربه آزاد نوعی چالش است که در فوتبال وجود دارد.
\subparagraph{نشان\protect\LTRfootnote{Badge}}
در ازای انجام چالش‌ها می‌توان به عنصر نشان رسید. وقتی بازیکنی کارهای خواسته‌شده را انجام می‌دهد، بازی به وی نشان افتخار می‌دهد \cite{atoz}. این نشان می‌تواند انگیزه بازیکن را برای ادامه بازی افزایش دهد؛ مثلا در فوتبال، دادن عنوان‌هایی مثل \emph{شوالیه} یا \emph{بیبی‌فیس قاتل} از سمت تماشاگران، نوعی از پیاده‌سازی این عنصر است.
\section{سناریوهای شکست}
اگر بازی‌وارسازی به خوبی پیاده و بازی به نحو مطلوب و اصولی طراحی نشود، شکست‌های بدی برای سازمان به وجود می‌آید. در ادامه چند سناریو از سایت آقای گیمیفیکیشن \cite{scenarios} به عنوان مثال‌هایی برای پیاده‌سازی‌های غلط بازی‌وارسازی آورده شده‌اند.
\subsection{بررسی وب‌سایت توسط متقاضی و اعطای نشان به وی}
مدیریت یک سایت تمایل دارد که متقاضیان استخدام با مطالعه سایت، اطلاعات بیشتری نسبت به محیط کار آینده خود و موقعیت شغلی که برای آن درخواست داده‌اند کسب کنند؛ به همین دلیل می‌خواهد از بازی‌وارسازی در فرایند استخدام استفاده کند تا متقاضیان را به مطالعه سایت ترغیب کند. پس سناریویی طراحی می‌کند که در آن متقاضی امتیازهایی برای مطالعه مطالب سایت به دست می‌آورد و بر اساس امتیازها به افراد نشان‌هایی اعطا می‌شود. به این شکل یک پیاده‌سازی از روش‌های بازی‌وارسازی ابداع می‌شود.

مشکل سناریو و پاداشی که داده می‌شود این است که با هدف کلی آن، یعنی استخدام، ارتباطی ندارد و نشان‌ها در این فرایند تاثیرگذار نیستند. به خاطر عدم وجود ارتباط بین هدف و پاداش، بازیکن رغبتی به انجام بازی ندارد و در نتیجه شناخت متقاضی نسبت به شرکت محقق نمی‌شوند.
\subsection{استفاده از سیستم معرفی برای ایجاد مشارکت در کارمندان و افزایش بهره‌وری}
مدیریت سازمان به دنبال ایجاد ارتباط بین کارمندان فعلی و کارمندان آینده سازمان است و همچنین می‌خواهد بین اعضای فعلی سازمان نیز مشارکتی ایجاد کند. برای همین از \emph{روش معرفی }استفاده می‌کند. روش معرفی سناریویی است که در آن اگر کارمند فعلی سازمان برای موقعیت‌های شغلی مورد نیاز مدیران، فردی را معرفی کند و فرد معرفی‌شده انتخاب شود، به او امتیازهایی داده می‌شود.

مشکل این سناریو این است که کار و بازی از یکدیگر جدا شده‌اند. در صورت پیاده‌سازی این سناریو، کارمندان ترجیح می‌دهند دائم موقعیت‌های شغلی را رصد کنند تا بتوانند برای هر موقعیت شغلی یک فرد مفید را پیشنهاد دهند و این یعنی بیشتر زمان کارمند صرف بازی معرفی می‌شود. از سمت دیگر توقعاتی برای کارمند ایجاد می‌شود؛ مثل این که کسانی که امتیاز بالایی دارند توقع دارند مزایایی مثل افزایش حقوق، پاداش و مرخصی‌های طولانی به آن‌ها تعلق گیرد که منطقی نیست ولی به خاطر اینکه نسبت به دیگران امتیاز بالایی دارند، در ذهن خود نتیجه‌گیری می‌کنند که از دیگر کارمندان سازمان بهتر هستند پس در نتیجه مستحق دریافت جوایز کلانی هستند.